\section{Moorer's Business}

Un successivo studio sulla simulazione digitale dei riverberi è stato condotto
da \jam~ intorno al 1979, pubblicando i suoi risultati su Computer Music Journal.
L’articolo, intitolato \emph{“About this Reverberation Business”}, tratta di
ulteriori integrazioni e accorgimenti che possono essere applicati durante lo
sviluppo di un riverbero.

L’autore, infatti, partendo dalla letteratura già presente, tra cui gli articoli
di Schroeder e J.Chowning, sperimenta nuovi algoritmi, rendendo più snelli i
precedenti, sempre tendendo ad una realisticità del riverbero.

Innanzitutto vengono ripresi i sistemi fondamentali dell’analisi di Schroeder,
vale a dire, il filtro Comb (visto in figura \ref{fig:dfl}) e il filtro
\emph{All-Pass} (visto in figura \ref{fig:apf}). Quest’ultimo in una seconda
versione caratterizzata da una singola moltiplicazione, a fronte delle 3
precedenti, rendendolo più sostenibile a livello di computazione.

\begin{figure}[htp]
\centering
\includegraphics[width=0.80\textwidth]{combmoorer}
\caption{Filtro Comb di Moorer}
\label{fig:combmoorer}
\includegraphics[width=0.80\textwidth]{apfmoorer}
\caption{Filtro All Pass di Moorer}
\label{fig:apfmoorer}
\end{figure}

Aventi le seguenti funzioni di trasferimento:
\begin{equation}
T(z) = \frac{g + z^{-m}}{1+gz^{-m}}
\end{equation}

\begin{equation}
T(z) = \frac{z^{-m}}{1-gz^{-m}}
\end{equation}

Osservando la funzione trasferimento dell’\emph{All-Pass} possiamo notare come
il coefficiente del numeratore sia in ordine inverso di quello al denominatore,
forzando gli zeri ad essere reciproci dei poli, definendo il comportamento
\emph{All-Pass} del filtro.

Seguendo il lavoro fatto da Schroeder, per utilizzare i sistemi come
riverberatori è necessario utilizzare una combinazione degli stessi.

Le combinazioni sono le medesime viste in precedenza e proposte da Schroeder
(fig \ref{fig:apfseq} e \ref{fig:comballpass}). Parliamo dunque di una serie di
All Pass nel primo algoritmo e, una cascata di comb filter seguiti da 2 All Pass
nel secondo

\begin{figure}[htp]
\centering
\includegraphics[width=0.80\textwidth]{allallpass}
\caption{Filtro Comb di Moorer}
\label{fig:allallpass}
\includegraphics[width=0.80\textwidth]{comballpassmoorer}
\caption{Filtro All Pass di Moorer}
\label{fig:comballpassmoorerB}
\end{figure}

%\subsection{Problematiche Del riverberatore}

Il riverberatore così ottenuto, però, non rispecchia alcune caratteristiche
desiderate, portando ad aberrazioni acustiche non presenti in natura.
Le problematiche riscontrate possono essere riassunte in:

\begin{itemize}
\item riverberazione non ottimale per suoni impulsivi e con transienti molto corti, producendo pattern ritmici composti dagli echi al posto di un riverbero uniforme;
\item riverberazione con un carattere metallico, soprattutto per tempi molto lunghi;
\end{itemize}

A questo punto l’autore considera l’utilizzo di nuove unità riverberanti, ma
comunque mantenendo la struttura Comb-All Pass, ritenuta la migliore in termini
di risposta.

%\subsection{Nuove unità riverberanti}

Nel corso delle sue sperimentazioni Moorer costruisce altre 4 unità riverberanti,
alcune molto simili tra di loro in quanto i concetti alla base restano i medesimi.
In particolare, l’intuizione dell’autore è stata quella di introdurre un ulteriore
filtro all’interno dei feedback.

Lo scopo del filtro, ricollegandoci agli argomenti trattati nei capitoli
precedenti, è quello di simulare l’attenuazione delle alte frequenze causate
dall’aria. Come già detto, è un coefficiente che, in base a caratteristiche
quali umidità e temperatura, sottrae energia alle alte frequenze dello spettro,
scurendo quest’ultimo.

%\subsection{Nuove unità riverberanti}

Come detto, le unità proposte sono 4, ma una in particolare sembra essere la
più efficiente, ovverosia:

\begin{figure}[htp]
\centering
\includegraphics[width=0.80\textwidth]{combfiltro}
\caption{Filtro Comb con all'interno un filtro Passa Basso}
\label{fig:combfiltro}
\end{figure}

Come è possibile notare, si tratta di un Comb filter al cui interno è presente
un filtro di tipologia Low pass $T(z)$. Il valore di $g_1$ controlla il roll-off
del filtro e in seguito troveremo un modo per capire quale valore sarà più
conveniente. All’interno, i valori $g_1$ e $g_2$ seguiranno la condizione
di $g_1+g_2<1$ per motivi di stabilità.

La risposta in frequenza, a detta di Moorer,  non sarà in grado di restituire
un valore coerente alla realtà, in quanto il singolo filtro Low pass (di primo
ordine) è un compromesso per rendere il tutto più efficiente.

Successivamente, nell’articolo viene anche consigliato di tenere in
considerazione della modifica dei tempi di delay in base alle caratteristiche
dell’aria (temperatura e umiditá), anche se nel testo si fa riferimento
valori standard e non modificabili.

I valori di $g$, dipendenti dall’umidità, sono in seguito esposti grazie agli
studi di Moorer nel seguente grafico.

\begin{figure}[h!]
\centering
\includegraphics[width=0.80\textwidth]{assorbimentomoorer}
\caption{Grafico raffigurante l'andamento del valore di $g$ a causa dell'assorbimento dell'aria}
\label{fig:assorbimentomoorer}
\end{figure}


%\subsection{In conclusione}

Moorer conclude la sezione riportando i risultati ottenuti. L’inserimento del
filtro permette una miglior riverberazione per i suoni impulsivi, infatti il
tempo di ogni eco risulta essere esteso, soprattutto per quanto riguarda le
prime riflessioni e nascondendo i vuoti creati dalla bassa densità.
Il risultato sonoro non è particolarmente entusiasmante a detta dell’autore,
ma sopperisce ad alcune mancanze delle precedenti iterazioni, rendendo questo
sistema il più efficace al momento della scrittura.
