% !TEX TS-program = pdflatex
% !TEX root = ../tesi.tex

%************************************************
\chapter{Filtri}
\label{chp:Filtri}
%************************************************

Questo capitolo affronta l'ampio mondo dei filtri, a quale scopo li utilizziamo, i loro comportamenti e architetture, ponendo uno sguardo all'utilizzo dei ritardi nel loro design.

\section{Filtri}
Prima di proseguire con l’analisi del riverberatore, c’è bisogno di introdurre il concetto di filtro, oggetto essenziale  per la creazione di quest’ultimo.
Un filtro possiamo definirlo come un dispositivo o rete di dispositivi in grado di separare segnali complessi in base alla loro frequenza. Un filtro, come suggerisce il nome, filtra e permette ad alcune bande di frequenze di essere successivamente emesse oppure escluse dall’output finale. 
Alcune caratteristiche presenti nella maggior parte dei filtri sono:
\begin{itemize}
\item Banda Passante:
Banda di frequenze che passa attraverso un filtro e che subisce una perdita di meno di 3 dB;
\item Banda Stoppata:
Banda di frequenze che passa attraverso un filtro e che subisce una perdita di 3 db o più; 
\item Frequenza di Taglio
È la frequenza dove il filtro inizia ad eseguire modifiche al segnale, come abbattimenti o enfatizzazioni;
\item Ordine:
È una caratteristica definita dal numero di oggetti all’interno del filtro che concorrono al medesimo risultato, di solito inerenti a scopi di risposta in frequenza del filtro. Se tutti gli elementi sono eterogenei, per esempio passa basso o passa alto, l’attenuazione sarà di 6 db per ordine. Un filtro passa basso del quarto ordine avrà abbattimento di 24 db, ma un passa banda del medesimo ordine ne abbatterà soltanto 12.
\end{itemize}
\medskip
\subsection{Categorie di filtri}
In seguito a queste caratteristiche, passiamo ad eseguire brevi categorizzazioni di filtri in base alla loro componentistica e comportamenti.

Una prima categorizzazione che possiamo eseguire, riguarda la presenza o meno, di componenti che amplificano il segnale. Parliamo di:
\begin{itemize}
\item Filtri Passivi:
Sono filtri che non possiedono amplificatori di segnale, quindi il loro unico scopo è quello di attenuare ciò che li attraversa;
\item Filtri Attivi:
Al contrario, i filtri attivi hanno componenti che permettono l’amplificazione di determinate bande o dell’intero segnale, aggiungendo energia dove necessario.
\end{itemize}

Parliamo di \textbf{equalizzatore} per identificare un dispositivo il cui scopo è quello di sopperire a caratteristiche non gradite, riguardanti ampiezza, frequenza o fase, in modo tale da ricreare la risposta desiderata. Gli equalizzatori, per compiere ciò, sono costituiti da filtri, implementati per svolgere diversi tipi di modifiche.

\section{Funzione di trasferimento}
Prima di proseguire è necessario affrontare in questa sezione un concetto matematico che negli anni ha contribuito allo studio e alla creazione di filtri digitali grazie alla sua estrema versatilità, ovvero le funzioni di trasferimento.
 
In breve, una funzione di trasferimento è la trasformata della risposta all’impulso di un sistema LTI (Linear Time-Invariant) e riassume in sé le sue caratteristiche.

Si presenta nella seguente forma:
\begin{equation}
H(s)=\frac{\sum_{k=0}^K a_k s^k}{\sum_{n=0}^N a_n s^n}
\end{equation}

È definita tramite una funzione razionale fratta, ovvero una funzione costituita dal rapporto di due polinomi. Ognuno dei due polinomi individua un’equazione rispettivamente di $k$-mo e $n$-mo grado ovvero che ci saranno $k$ soluzioni per il polinomio al numeratore ed $n$ soluzioni per il polinomio al denominatore. Queste soluzioni saranno tutti i valori che rendono nulla la funzione: quelle che azzerano il numeratore si dicono \textbf{zeri} $(-zk)$, quelle che azzerano il denominatore si dicono \textbf{poli} $(-pn)$.

La forma fattorizzata della funzione di trasferimento è la seguente ed è utilizzata per il tracciamento dei grafici di risposta in frequenza dei filtri (detti diagrammi di bode)
\begin{equation}
H(s)=\frac{\prod_{k=1}^k (s+zk)}{\prod_{n=1}^n (s+pn)}
\end{equation}
\bigskip 
Semplificando il tutto e ricollegandoci all’argomento principale, è necessario comprendere che poli e zeri corrispondono alla frequenza di taglio del filtro che la funzione di trasferimento rappresenta.
\section{Tipologie di filtro}
Come detto in precedenza, i filtri si distinguono per via della loro architettura alla quale consegue un diverso comportamento. Ecco alcune tipologie di filtro, tra le più comuni:

\subsection{Low-Pass}
Permette l’attenuazione delle frequenze superiori alla frequenza di taglio.
La sua funzione di trasferimento è la seguente
\begin{equation}
H(s)=\frac{1}{1+st}
\end{equation}
dove con $st$ indichiamo la frequenza di taglio.

\begin{figure}[h]
\centering
\includegraphics[width=%
0.50\textwidth]{lp}
\caption{Diagramma di Bode di un filtro passa basso \newline \scriptsize{ da D. Davis - Sound System Engineering - 2013}}
\label{fig:lp}
\end{figure}

\subsection{High-Pass}
Permette l’attenuazione delle frequenze inferiori alla frequenza di taglio.
La sua funzione di trasferimento è in figura \ref{fig:hp}

\begin{equation}
H(s)=\frac{st}{1+st}
\end{equation}

dove con $st$ indichiamo sempre la frequenza di taglio.
La risposta in frquenza ottenuta è la seguente
\begin{figure}[htp]
\centering
\includegraphics[width=%
0.50\textwidth]{hp}
\caption{Diagramma di Bode di un filtro passa alto}
\label{fig:hp}
\end{figure}

\subsection{Band-Pass}
Permette il passaggio solo ad una determinata banda di frequenze.
la sua funzione di trasferimento è la seguente
\begin{equation}
H(s)=\frac{(1+st_1)(1+st_4)}{(1+st_2)(1+st_3)}
\end{equation}
Considerando che $t_1>t_2>t3_>t_4$ e che il calcolo di poli e zeri risulta: $z_1=1/t_1$, $z_2=1/t_4$, $p1_=1/t_2$, $p_2=1/t_3$ \dots

La risposta in frquenza ottenuta è in figura \ref{fig:bp}
\begin{figure}[htp]
\centering
\includegraphics[width=%
0.50\textwidth]{bp}
\caption{Diagramma di Bode di un filtro passa banda}
\label{fig:bp}
\end{figure}


\section{Filtri Digitali}

Ovviamente tutto ciò che abbiamo visto fino a questo punto è puramente nel dominio continuo. Possiamo dunque procedere all’implementazione di tali modelli matematici in digitale. I filtri digitali utilizzano in larga scala algoritmi ricorsivi aventi al loro interno moltiplicazioni e addizioni, per i quali gli attuali strumenti informatici di cui disponiamo, sono ottimizzati. Definiamo 2 macro categorie di filtri digitali ovvero i cosiddetti filtri \textbf{FIR} e \textbf{IIR}.
\begin{itemize}
\item IIR: Infinite impulse response;
\item FIR: Finite impulse response.
\end{itemize}
Possiamo definire i filtri IIR come caratterizzati da una risposta all’impulso unitario non limitata (parlando di numero di campioni) e un’ampiezza tendente a zero.
Questo rende i filtri IIR molto simili alla loro controparte analogica (esistente nel tempo continuo).
Per quanto riguarda la loro composizione strutturale, notiamo generalmente una funzione di trasferimento costituita da un rapporto polinomiale, con poli e zeri. È esclusa una situazione in cui siano presenti soli zeri. Inoltre la costruzione di questi filtri prevede l’utilizzo di sistemi di feedback, e per questo, è risaputo che queste strutture hanno un comportamento instabile. 

Parlando di filtri FIR, invece, possiamo affermare che la loro risposta all’impulso unitario è composta da un numero finito di campioni e per questo non associabili a nessun modello analogico.
A differenza dei filtri IIR, i filtri FIR, presentano una funzione di trasferimento a soli zeri.
Possedere una risposta all’impulso unitario limitata comporta, inoltre, una stabilità nell'implementazione, ma di conseguenza una necessità di maggiore potenza di calcolo.
(Lindoro del Duca - Musica Digitale - pag 85)

\section{Delay e Utilizzo}
Il \emph{Delay} è il ritardo temporale imposto ad un evento sonoro. Il Delay può essere percepito soltanto se messo in relazione ad un altro segnale, ovvero, il suono stesso ma non ritardato.
Questo fenomeno si manifesta in due situazioni: quando, in un ambiente riverberante il suono diretto raggiunge l’ascoltatore prima delle sue riflessioni, oppure quando, un delay elettrico, è volontariamente inserito per pareggiare due sorgenti sonore e simulare un’ unica provenienza.

Il delay è inoltre un componente chiave nella costruzione dei filtri digitali.
Nel dominio digitale, infatti, i singoli campioni di una sequenza vengono sottoposti a numerose operazioni (implicando un processo di quantizzazione) producendo una seconda sequenza di risultati che comporranno il segnale filtrato. 
Le operazioni utilizzate sono: Somma, Prodotto e Delay (di un campione).
L’espressione che determina il comportamento del filtro è detta equazione differenza e, in base al numero di elementi di ritardo presenti, possiamo indicare l’ordine dell’equazione.

L’equazione differenza del primo ordine generale è definita secondo la seguente equazione:

\begin{equation}
Y(z)=A_0*x(z) + A_1*z^{-1}*x(z) + B_1*z^{-1}*y(z)
\end{equation}

L’equazione è scritta nel dominio z, con z detta frequenza complessa.
I valori di z che rendono rispettivamente nullo numeratore e denominatore vengono detti zeri e poli della funzione.

Tutte le equazioni differenza del primo ordine sono determinate dai coefficienti $A_0, A_1,B_1$.
Bisogna sempre considerare che un’equazione del primo ordine può descrivere solo un filtro passa basso o passa alto avente un’attenuazione di massimo 6 dB/ottava.

Partendo dall’equazione differenza, infine, possiamo ricavare la funzione di trasferimento $H(z)$ avente la seguente forma:

\begin{equation}
H(z)=\frac{(A_0 + A_1 * z^{-1})} {(1-B_1 * z^{-1})}
\end{equation}
