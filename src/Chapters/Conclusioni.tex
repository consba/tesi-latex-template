% !TEX TS-program = pdflatex
% !TEX root = ../tesi.tex

%************************************************
\chapter{Conclusioni}
\label{chp:Conclusioni}
%************************************************
Sulla base dei risultati ottenuti nel capitolo \ref{chp:Implementazione}, possiamo trarre alcune
considerazioni sul lavoro svolto.
Certo, si può affermare senza dubbio che le condizioni atmosferiche hanno una certa
influenza sul risultato riverberante; sia il tempo che il carattere di riverberazione, infatti, hanno un
risultato differente in base ai parametri che si vanno man mano modulando. I risultati sonori
non sono certo eccezionali, se consideriamo il tutto da un punto di vista prettamente scientifico.
Certamente non ci troviamo di fronte ad una rappresentazione fedele della realtà date le numerose
approssimazioni ma, dal punto di vista espressivo, delle potenzialità mi sento di vedercele. A fronte
del lavoro svloto, dei risultati e in generale dell'esperienza ottenuta svolgendo questa tesi,
posso tranquillamente affermare di trovarmi di fronte un buon punto di partenza per ulteriori
miglioramenti e approcci diversificati per raggiungere l'obiettivo.

\todo[inline]{credo tu abbia confuso il ruolo del paragrafo conclusivo. non si
tratta di dare giudizi soggettivi o considerazioni personali. Si trtta di riprendere
gli argomenti aperti con l'introduzione e concludere il cerchio spiegando cosa
ha porta a cosa e con quali caratteristiche e corrispondenze. Per esempio:

il presupposto di controllare un riverbero artificiale ha dato risultati difficilmente
ottenibili con i parametri tradizionali?

il controllo parametrico ambientale è qualcosa che fa relazionare con lo strumento
riverbero in modo diverso dai parametri tradizionali? perchè?

inoltre, dato che tu non inventi un nuovo riverbero ma utilizzi riverberi storici
per mettere a punto la matematica ambientale, non ha senso parlare di bellezza del
riverbero}
