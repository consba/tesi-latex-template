\section{Wallace C. Sabine: l'alba dell'acustica architettonica}

\todo[inline]{da qui: ho inserito pari pari il testo preso da METP riguardante
la ricerca di Sabine. valuta se tenere delle cose che ti interessano e rielaborale
facendole tue. solo se c'è qualcosa che ti interessa.}

\bigskip

Nonostante l'acustica architettonica sia stata per millenni parte integrante
delle capacità progettuali di strutture, la \emph{fisica acustica} ha ottenuto
una base scientifica solida solo nei primi del novecento, ad opera di
\ws\footcite{ws:rev}. La definizione del tempo di riverbero da parte di Sabine è un
punto di partenza anche nella letteratura sulla modellazione digitale
dell'effetto, fin dal principio, nell'opera di \ms\footcite{ms:rev62, ms:rev64}.
Dopo Sabine il tempo di riverbero può essere descritto, misurato, previsto.
Tutto quello che sappiamo fare oggi continua ad attingere alle sue ricerce.

\begin{quote}
  The following investigation was not undertaken at first by choice, but devolved
  on the writer in 1895, through instructions from the Corporation of Harvard
  University to propose changes from remedying the acoustical difficulties in
  the lecture-room o the Fogg Art Museum, a building that had just been completed.
  About two years were spent in experimenting on this room, and permanent changes
  where then made. Almost immediately afterward it become certain that a new
  Boston Music Hall would be erected, and the questions arising in the
  consideration of its plans forced a not unwelcome continuance of the general
  investigation\footcite{ws:rev}.
\end{quote}

Trovo significativo che una ricerca così approfondita e miliare possa essere
scaturita da una semplice problematica come quella di dover analizzare e
correggere l'acustica di una sala universitaria. Sulla base di un vuoto letterario,
nulla di organico sul fenomeno del riverbero se non per cenni sparsi provenienti
dalla storia della fisica, \ms~ ha costruito una ricerca organica, con seri
problemi da risolvere, anche di diversa natura, come per esempio la scelta del
cronografo (1885) per misurare il tempo

\begin{quote}
  \ldots perfect noiselessness, portability, and capacity to measure intervals
  of time from a half of second to ten seconds with considerable accuracy\footcite{ws:rev}.
\end{quote}

Sono proprio queste parole a rivelare che non poteva essereci un momento diverso
nella storia dell'uomo in cui la convergenza di esigenze e possibilità tecniche
avrebbe portato alla soluzione di problematiche irrisolte (o non considerate)
per secoli.

\begin{quote}
  In order that hearing may be good in any auditorium, it is necessary that the
  sound should be sufficiently loud; that the simultaneous components of a
  complex sound should maintain their proper relative intensities;
  and that the successive sounds in rapidly moving articulation, either of speech
  or music, should be clear and distinct, free from each other and from extraneous
  noises. Thesethree are the necessary, as they are the entirely sufficient,
  conditions for good hearing\footcite{ws:rev}.
\end{quote}

Una tripletta di problemi minimi da comprendere e risolvere per rendere accettabile
il riverbero acustico di un ambiente.

\subsection{Loudness}

Illustrando una condizione semplice di auditorium in forma di spazio piano,
con un oratore ed un ascoltatore, \ws~ introduce il concetto di propagazione
del suono in forma emisferica, che si riduce di intensità all'aumentare della
sua dimensione (distanza dall'origine) proporzionalmente. Aumenta il pubblico,
il suono perde intensità più rapidamente, assorbito. La parte superiore della
propagazione si muove libera, non affetta da assorbimenti. I primi accorgimenti:
elevare l'oratore ed alzare da terra le file posteriori: il teatro Greco. Un tetto
su questa struttura incrementerebbe l'intensità media, soprattutto dei suoni
sostenuti nel tempo, e ne bilancerebbe la resa tra fronte e fondo sala.

\begin{quote}
  The problem of calculating the loudness at different parts of such an
  auditorium is, obviously, complex, but it is perfectly determinate, and as
  soon as the reflecting and absorbing power of the audience and of the various
  wall-surfaces are known it can be solved approximately\footcite{ws:rev}.
\end{quote}

Ne ricaviamo la prima ufficiale considerazione: non si può parlare di Riverbero,
al singolare, ma di Riverberi di un luogo. Perfettamente determinati, calcolabili,
ma molti per ogni ambiente che descriviamo.

\subsection{Distortion of Complex Sounds: Interference and Resonance}

In termini di \emph{loudness}, i suoni diretti ed i suoni riflessi si rinforzano
l'un laltro quando viaggiano insieme. Possono però trovarsi nella condizione di
cancellarsi a vicenda. Nella descrizione del ronte d'onda che si muove per
successioni di stati opposti, compressioni e rarefazioni, si possono avere
condizioni in cui il suono riflesso da pareti distinte produca nello spazio della
sala una zona di incontro di queste riflessioni, in cui entrambe le compressioni e
le rarefazioni si trovino rinforzate, in fase. Ma si può avere l'occorrenza opposta,
un punto di incontro in cui le compressioni e le rarefazioni non si succedono ma
si sovrappongono, annullandosi. Tutto questo accade in relazione al suono emesso,
alla sua altezza, che variando, varia l'intero stato di equilibrio, l'inntero stato
di interferenza.

C'è un altro fenomeno che occorre in queste circostanze, in relazione con
l'intererenza, ovvero la risonanza.

\begin{quote}
  The word \emph{resonance} has been used loosely as synonymous with
  \emph{reverberation}, and even with \emph{echo}, and is so given in some of
  the more voluminous but less exact popular dictionaries. In scientific
  literature the term has received a very definite and precise application to
  the phenomenon, wherever it may occur, of the growth of a vibratory motion of
  an elastic body under periodic force stimed to its natural rates of vibration.
  A word having this significance is necessary; and it is very desirable that
  the term should not, even popularly, by meaning many things, cease to mean
  anything exactly\footcite{ws:rev}.
\end{quote}

Gli uomini che chiedono rispetto per le parole, meritano rispetto, perché
rispettano gli uomini. Anche questa è risonanza.

\subsection{Confusion: Reverberation, Echo and Extraneous Sounds}

Si entra così nel cuore della ricerca di \ws, l'atto pratico di comprendere il
malfunzionamento acustico del luogo speciico, cinque secondi ed oltre di riverbero
tale da rendere impossibile comprendere la propria voce in una semplice discussione.
Il fenomeno definito \rev~ il processo delle rilessioni multiple,
tra superfici, pareti, soffitto e pavimento, dapprima da una e poi da un'altra e
poi da molte superici, cambiando (o perdendo) un poco ad ogni rilessione, fino
a diventare inudibile. Questo il fenomeno \rev, che include il caso
speciale denominato \eco. Il \rev~ consiste inoltre in una massa
di suono che riempie uno spazio della quale è impossibile cogliere ed analizzare
la singola riflessione. Il termine \eco~ è riservato al caso specifico di
riflessione pulita, singola, generata da una singola superficie ed a volte ripetuta,
nel caso di più superfici riflettenti. Nel \rev~ ci concentriamo a definirne il
tasso di decadimento del suono nel tempo, nel caso dell'\eco~ l'intensità è un
afttore secondario, mentre risulta un fattore discriminante l'intervallo temporale
tra il suono originario ed il tempo di arrivo della riflessione all'ascoltatore.

La misurazione temporale diventa quindi fondamentale, oggi piuttosto scontata
per misurazioni fisiche di ordine infinitamente piccole, ma per \ws~ non era proprio
così.

Il percorso di misurazione del tempo di decadimento del \emph{suono residuo} (il
suono che resta in aria dopo che la fonte sonora ha cessato di produrlo) evidenzia
a \ws~che ci sono due e due variabili soltanto di un luogo  ad influire sul
risultato cronometrico: la forma della stanza, inclusa la dimensione; i materiali,
incluso l'arredamento.

\subsection{Echi di Sabine}

Ci sono innumerevoli spunti di riflessione tra le pagine dei testi di \ws\footcite{ws:rev},
dai quali, agli scopi di una corretta implementazione digitale del riverbero e
soprattutto agli scopi di un corretto utilizzo musicale, possiamo ricavare:

\begin{compactitem}
  \item La durata del suono residuo ascoltabile in un ambiente è approssimativamente
  uguale in ogni punto dello spazio.
  \item La durata del suono residuo ascoltabile in un ambiente è approssimativamente
  indipendente dalla posizione della sorgente.
\end{compactitem}

Sono questi due presupposti fondamentali, sui quali cercheremo di costruire un
pensiero musicale prima ancora che uno strumento musicale, quale il riverbero
digitale può essere.

\bigskip

\todo[inline]{a qui}



%Tappa fondamentale riguardante lo studio sui riverberi è la ricerca condotta ad Harvard da Wallace C. Sabine, considerato il padre dell’acustica ambientale.
%Nel suo articolo, l’autore affronta le problematiche legate all’acustica e alla riverberazione negli auditorium e in generale spiega le condizioni che permettono una buona udibilitá sia del parlato che della musica.
%L’articolo è frutto di una serie di esperimenti condotti nella stanza dedicata alla lettura del Foggy Art Museum. Per ben 2 anni, infatti, Sabine e i suoi collaboratori hanno eseguito cambiamenti architettonici nell’aula in modo da migliorane le condizioni acustiche bistrattate dagli studenti che ne usufruivano.

Grazie a queste ricerche, numerosi progressi sono stati fatti nell’ambito dell’acustica ambientale e consistono nelle fondamenta delle attuali conoscenze.
Innanzitutto l’autore identifica le condizioni necessarie per una buona acustica:
\begin{compactitem}
\item Intensità adeguata del suono;
\item Distorsione minima dell’onda complessa;
\item Percezione chiara delle riflessioni.
\end{compactitem}

Gli studi condotti, ponendo questi come gli obiettivi da raggiungere, hanno riscontrato come maggiori cause cattive riverberazioni, risonanze e assorbimento acustico non ottimale.

In generale gli esperimenti si concentrano sulla durata del decadimento del suono riverberato e dell’influenza che hanno pareti e corpi all’interno della stanza sull’assorbimento dell’energia generale.

Il culmine di questo studio, oltre a dimostrare che esiste una correlazione tra la quantità di superficie assorbente (pareti, sedute, persone) e la qualità di percepimento del suono in una stanza, è lo sviluppo di una formula in grado di ricavare il tempo in cui il suono decade fino ad una situazione di equilibrio.
Parliamo di \emph{RT60} ovvero il tempo in cui il suono (riverberato) decade di 60 dB.

\begin{equation}
RT60 = \frac{24(\ln{10})V}{c s_a}
\end{equation}

Di cui:
\begin{compactitem}
\item $V$ è il volume della stanza
\item $c$ è la velocità del suono
\item $s_a$ è il valore di assorbimento totale espresso in Sabins
\end{compactitem}

Possiamo calcolare i Sabins sommando l’area totale delle pareti (per esempio 4 mura + 1 soffitto e 1 pavimento) e moltiplicandola per il coefficiente di assorbimento (ovviamente il coefficiente può essere diversificato per i diversi materiali delle pareti). Da notare che il coefficiente è un valore che varia tra 0 (minimo assorbimento) e 1 (massimo assorbimento).

L’articolo è considerato un capolavoro di acustica applicata e ha avuto una grande influenza sullo sviluppo della scienza del suono e sulla progettazione degli spazi sonori.
