\section{Architetture di riverberazione artificiale}

Non esiste un unico modo di ricreare un riverbero ma, come è possibile intuire, negli anni si sono sviluppate differenti tecniche per raggiungere questo scopo.
Innanzitutto bisogna definire 2 categorie di principali:

\begin{itemize}
\item Analogici
\item Digitali
\end{itemize}

Le tecniche di riverberazione analogica, non presentano processi di trasformazione digitale del segnale, senza quindi utilizzare operazioni matematiche al loro interno.
Di questa categoria citiamo:

\begin{itemize}
\item Riverberi Elettromeccanici: Questa tipologia di riverbero utilizza un elemento riverberante all’interno del proprio circuito. Due esempi degni di nota sono gli \emph{“spring reverb” e “plate reverb”} che, come suggerisce il nome, utilizzano molle, nel primo caso e placche metalliche, nel secondo, per simulare l’effetto del riverbero sul segnale originale. In poche parole l’elemento riverberante fungeva da ponte tra l’entrata e l’uscita del sistema, modificando le proprietà acustiche del segnale in input.
\item Camere riverberanti: Questa tipologia si serve di uno spazio realmente esistente al cui interno è presente un diffusore ed un microfono. Il segnale originale viene emesso dall’altoparlante che, diffondendosi nello spazio circostante, acquisisce un riverbero non presente all’origine. Il risultato viene poi successivamente registrato dal microfono, conservando le nuove proprietà.
\end{itemize}

Per quanto riguarda le tecniche di \emph{riverberazione digitale}, parliamo di processi in cui il segnale originale, digitalizzato, subisce modifiche tramite calcoli matematici. La tecnica più diffusa è quella del riverbero algoritmico che prevede una serie di \textit{somme, prodotti e delay}. Successivamente parleremo in modo più dettagliato di questa tecnica.

Degna di nota è un’ulteriore tecnica digitale, diffusasi negli ultimi anni, che utilizza la \emph{convoluzione}. Senza entrare troppo nei dettagli la convoluzione è un processo matematico di moltiplicazione tra due segnali che avviene campione per campione. La convoluzione è utilizzata tra il segnale originale e la risposta all’impulso di uno spazio esistente, producendo in output un segnale avente le medesime caratteristiche di quest’ultimo.
