% !TEX TS-program = pdflatex
% !TEX root = ../tesi.tex
%************************************************
\chapter*{Introduzione}
\label{chp:Introduzione}
%************************************************

In questo lavoro di tesi affronto argomenti di riverberazione artificiale e
come l'ambiente riverberante influisce sulla percezione sonora. In particolare
la mia ricerca si è concentrata sullo studio del comportamento delle vibrazioni
acustiche al variare delle \emph{condizioni atmosferiche}, in particolare nelle
qualità di: temperatura, umidità e pressione atmosferica; in presenza di mezzi
di trasmissione acustica diversi dall’aria. Lo studio dei fenomeni fisici è
confluito nell'implementazione di un riverbero a parametri di controllo
atmosferici.

La tesi attinge alle ricerche princicpali che nel novecento hanno introdotto il
concetto di riverberazione artificiale (Schroeder) e poi sviluppato quelle
tecniche (Moorer) contribuendo allo sviluppo delle possibilità di simulazione di
un ambiente sonoro realistico.

Quando ci troviamo in un determinato luogo, che sia un appartamento, un ufficio o altro, non sempre facciamo caso alle sue proprietà acustiche, magari scorgiamo altri dettagli, come una certa corrispondenza tra i colori delle pareti e gli oggetti di arredamento, ma, le relazioni che intercorrono tra l’ambiente e la percezione sonora, sono informazioni che spesso trascuriamo o che addirittura risultano superflue.

In un ambiente reale, i rapporti tra gli elementi presenti al suo interno, sono innumerevoli, definendo in maniera quasi assoluta la peculiarità dei suoni che si propagano. Possiamo pressocchè dire che l’evento acustico è legato indissolubilmente allo spazio che ha attorno. La tesi ha lo scopo di indagare l’influenza che le caratteristiche ambientali e climatiche hanno sulla propagazione di un’ onda sonora, portando allo studio e implementazione di algoritmi in grado di simulare queste ultime. Lo sviluppo di questa tesi è stato guidato principalmente dalla curiosità dell’autore di comprendere se e quanto, queste caratteristiche esterne all’evento sonoro, possano definirne il timbro e la percezione.
Gli argomenti affrontati nei capitoli successivi, serviranno a gettare le basi teoriche per l’implementazione finale, ovvero un sistema informatico in grado di simulare uno spazio nel quale è possibile effettuare modifiche sia strutturali che relative al clima.
Nel primo capitolo è presente un’introduzione al mondo dell’acustica, con un focus sulle formule matematiche che permettono di esplorare i vari fenomeni partecipi alla produzione e diffusione di un onda sonora.

Successivamente viene esplorato l’aspetto relativo agli oggetti matematici essenziali per la creazione di un riverbero sintetico. Il secondo capitolo, infatti, si occupa di introdurre il concetto di filtro insieme ai concetti matematici di somme e ritardi, utilizzati per l’approccio informatico ai filtri.
Infine gli ultimi due capitoli sono interamente dedicati ai riverberi. All’interno del terzo capitolo, oltre all’introduzione delle tecniche sviluppate negli anni per la simulazione di un riverbero, sono analizzati i lavori di 3 importanti studiosi del secolo scorso, ovvero Sabine, Schroeder e Moorer, i quali hanno fornito una solida base per l’implementazione dei sistemi. Il quarto ed ultimo capitolo ha invece il compito di presentare il lavoro implementato, mostrando il codice scritto nel linguaggio Faust.
