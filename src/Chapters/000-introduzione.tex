% !TEX TS-program = pdflatex
% !TEX root = ../tesi.tex
%************************************************
\chapter*{Introduzione}
\addcontentsline{toc}{chapter}{INTRODUZIONE}
\label{chp:Introduzione}
%************************************************

%\todo[inline]{ho sistemato un po' il testo, sempre nell'ottica di andare a
% eliminare eventuali fragilità che lasciano aperte troppe questioni di fisica pura.
% manca però una terza parte dell'introduzione in cui spieghi quali sono le parti
% della tesi con i nomi delle parti: es. nel capitolo 1, storia, bla bla bla; nel
% capitolo 2 ciccio\ldots bla bla bla}

Quando ci troviamo in un determinato luogo, che sia un appartamento, un ufficio o altro, non sempre 
facciamo caso alle sue proprietà acustiche, magari scorgiamo altri dettagli, come una certa 
corrispondenza tra i colori delle pareti e gli oggetti di arredamento, ma, le relazioni che 
intercorrono tra l’ambiente e la percezione sonora, sono informazioni che spesso trascuriamo 
o che addirittura risultano superflue.
In un ambiente reale, i rapporti tra gli elementi presenti al suo interno, sono innumerevoli, 
definendo in maniera quasi assoluta la peculiarità dei suoni che si propagano. 
Possiamo pressocchè dire che l’evento acustico è legato indissolubilmente allo spazio che ha attorno.

In questo lavoro di tesi affronto argomenti di riverberazione artificiale e
come l'ambiente riverberante influisce sulla percezione sonora. In particolare
la mia ricerca si è concentrata sullo studio del comportamento delle vibrazioni
acustiche al variare delle \emph{condizioni atmosferiche}, in particolare nelle
qualità di: temperatura, umidità e pressione atmosferica; in presenza di mezzi
di trasmissione acustica diversi dall’aria. Lo studio dei fenomeni fisici è
confluito nell'implementazione di un riverbero a parametri di controllo
atmosferici.

La tesi attinge alle ricerche princicpali che nel novecento hanno introdotto il
concetto di riverberazione (Sabine), tecniche di riverberazione artificiale
(Schroeder) e poi elaborato quelle tecniche (Moorer) contribuendo allo sviluppo
delle possibilità di simulazione di un ambiente sonoro realistico.

Il testo è diviso in tre parti: la presente introduzione, una parte storica, le
strategie di implementazione che ho adottato.
All'interno della parte storica mi sono occupato di raccogliere informazioni e organizzarle, 
cercando di tracciare un filo conduttore in grado di ripercorrere i vari stadi dello studio 
dei riverberi. Partendo dagli studi di \ws, fondamenta della fisica acustica, ho poi
esposto brevemente alcune nozioni di carattere fisico e matematico (senza addentrarmi nei dettagli) 
essenziali per l'implementazione del riverbero, concludendo infine esplorando l'ambito dei riverberi
sintetici, concentrandomi sull'analisi degli scritti di \ms~ e Moorer.

All'interno della parte implementativa tutto ciò che è stato esposto precedentemente viene
riutilizzato in favore della realizzazione del riverbero Atmoverb. I codici presenti sono scritti in
linguaggio \faust~ e ricostruiscono (in digitale) il lavoro eseguito durante il secolo scorso da
\ms~ e James Moorer.

Lo sviluppo di questa tesi è stato guidato principalmente dalla curiosità 
dell’autore di comprendere se e quanto, queste caratteristiche esterne all’evento sonoro, possano 
definirne il timbro e la percezione.
%\todo{descrivere le parti}

\bigskip

% \todo[inline]{i periodi successivi non fanno parte dell'introduzione.
% anddrebbero sistemati ma lo farei dopo che gli hai trovato un posto.}

% Quando ci troviamo in un determinato luogo, che sia un appartamento, un ufficio o altro, non sempre facciamo caso alle sue proprietà acustiche, magari scorgiamo altri dettagli, come una certa corrispondenza tra i colori delle pareti e gli oggetti di arredamento, ma, le relazioni che intercorrono tra l’ambiente e la percezione sonora, sono informazioni che spesso trascuriamo o che addirittura risultano superflue.

% In un ambiente reale, i rapporti tra gli elementi presenti al suo interno, sono innumerevoli, definendo in maniera quasi assoluta la peculiarità dei suoni che si propagano. Possiamo pressocchè dire che l’evento acustico è legato indissolubilmente allo spazio che ha attorno. La tesi ha lo scopo di indagare l’influenza che le caratteristiche ambientali e climatiche hanno sulla propagazione di un’ onda sonora, portando allo studio e implementazione di algoritmi in grado di simulare queste ultime. Lo sviluppo di questa tesi è stato guidato principalmente dalla curiosità dell’autore di comprendere se e quanto, queste caratteristiche esterne all’evento sonoro, possano definirne il timbro e la percezione.
